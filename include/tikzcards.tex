%   COMMANDS ZUM ZUSAMMENBAUEN DER KARTEN
%   ---------------------------------------

%   TikZ/PGF Settings für die Karten

% A6 dimentions: 105 x 148mm 
% -page margins: 95 x 138mm

\pgfmathsetmacro{\cardwidth}{9}
\pgfmathsetmacro{\cardheight}{13}
\pgfmathsetmacro{\imagewidth}{\cardwidth}
\pgfmathsetmacro{\imageheight}{0.75*\cardheight}
\pgfmathsetmacro{\stripwidth}{0.7}
\pgfmathsetmacro{\strippadding}{0.2}
\pgfmathsetmacro{\textpadding}{0.1}
\pgfmathsetmacro{\titley}{\cardheight-\strippadding-1.5*\textpadding-0.5*\stripwidth}
% For Catan
\pgfmathsetmacro{\titleheight}{2*\stripwidth}
\pgfmathsetmacro{\resourcewidth}{\cardwidth-2*\stripwidth-2*\strippadding}
\pgfmathsetmacro{\RuleImgwidth}{\cardwidth-4*\strippadding-2*\stripwidth -2*\strippadding}
\pgfmathsetmacro{\RuleImgheight}{\cardheight-0.31*\cardheight-2*\stripwidth-2*\strippadding  -2*\strippadding}
\pgfmathsetmacro{\miniwidth}{\stripwidth-0.1}


%   Forms of the individual map elements / components
\def\shapeCard{(0,0) rectangle (\cardwidth,\cardheight)}

\def\shapeLeftStripLong{(\strippadding,-0.2) rectangle (\strippadding+\stripwidth,\cardheight-\strippadding-\strippadding-1)}

\def\shapeLeftStripShort{(\strippadding,\cardheight-\strippadding-1) rectangle (\strippadding+\stripwidth,\cardheight+0.2)}

\def\shapeRightStripShort{(\cardwidth-\stripwidth-\strippadding,\cardheight-\strippadding-1) rectangle (\cardwidth-\strippadding,\cardheight+0.2)}

\def\shapeTitleArea{(2*\strippadding+\stripwidth,\cardheight-\strippadding) rectangle (\cardwidth-2*\strippadding-\stripwidth,\cardheight-\titleheight)}

\def\shapeContentArea{(2*\strippadding+\stripwidth,0.5*\cardheight) rectangle (\cardwidth+0.2,-0.2)}

% Custom catan shapes

\def\shapeDescriptorArea{(-0.2,   0.31*\cardheight) rectangle (\cardwidth+0.2,-0.2)}

% Deprecated: moved to another location
\def\shapeExpansionArea{(\cardwidth-\stripwidth-5*\strippadding,\cardheight-\strippadding) rectangle (\cardwidth-\strippadding,\cardheight-\titleheight)}

\def\shapeLeftStripExtra{(\strippadding,\strippadding) rectangle (\strippadding+\stripwidth,\cardheight-\strippadding)}

\def\shapeRuleImg{(2*\strippadding+\stripwidth,0.31*\cardheight+\strippadding) rectangle (\cardwidth-2*\strippadding-\stripwidth,\cardheight-\titleheight-\strippadding)}

\def\shapeResourceMiniUL{(\strippadding,\cardheight-\titleheight) rectangle (\strippadding+\stripwidth,\cardheight-\strippadding)}

\def\shapeResourceMiniUR{(\cardwidth-\stripwidth-\strippadding,\cardheight-\titleheight) rectangle (\cardwidth-\strippadding,\cardheight-\strippadding)}

\def\shapeResourceMiniBL{(\strippadding,\strippadding) rectangle (\strippadding+\stripwidth,\titleheight)}

\def\shapeResourceMiniBR{(\cardwidth-\stripwidth-\strippadding,\strippadding) rectangle (\cardwidth-\strippadding,\titleheight)}


%   Define styles for the elements
\tikzset{
    %   round corners for the cards
    cardcorners/.style={
        rounded corners=0.2cm
    },
    %   round corners for the "flags"
    elementcorners/.style={
        rounded corners=0.1cm
    },
    %   Drop shadow for the "flags"
    stripshadow/.style={
        drop shadow={
            opacity=.5,
%            shadow,
            color=black
        }
    },
    %   Style für die "Fähnchen"
    strip/.style={
        elementcorners,
        stripshadow
    },
    %   Bild für das Kartenmotiv
    cardimage/.style={
        path picture={
            \node[below=15mm] at (0.5*\cardwidth,\cardheight) {
                \includegraphics[width=\imagewidth cm]{#1}
            };
        }
    },
    %% Catan
    centerImg/.style={
		path picture={
			\node[below=20mm] at (0.5*\cardwidth,\cardheight) {
				\includegraphics[width=\resourcewidth cm]{#1}
			};
		}
	},
	ruleImage/.style={
        path picture={
            \node at (path picture bounding box.center) {
                \includegraphics[
                    height=\RuleImgheight cm,
                    width=\RuleImgwidth cm,
                    % height=2 cm,
                    % width=2 cm,
                    keepaspectratio]{#1}
            };
        }
    },
    miniImage/.style={
        path picture={
            \node at (path picture bounding box.center) {
                \includegraphics[
                % titleheight is not entierly correct, since it does not include strippadding
                    height=\titleheight cm,
                    width=\miniwidth cm,
                    keepaspectratio]{#1}
            };
        }
    },
    miniFlipImage/.style={
        path picture={
            \node at (path picture bounding box.center) {
                \includegraphics[
                % titleheight is not entierly correct, since it does not include strippadding
                    height=\titleheight cm,
                    width=\miniwidth cm,
                    keepaspectratio,
                    angle=180]{#1}
            };
        }
    },
}

%   TikZ-Raster
\newcommand{\carddebug}{
    \draw [step=1,help lines] (0,0) grid (\cardwidth,\cardheight);
}

%   Rahmen der Karte
\newcommand{\purecardborder}{
    \draw[lightgray,thick,cardcorners] \shapeCard;
}

%   Hintergrund der Karte
\newcommand{\cardbackground}[1]{
    \draw[cardcorners, cardimage=#1] \shapeCard;
}

%% Catan
\newcommand{\devImg}[1]{
    \path[centerImg=#1] \shapeCard;
}

\newcommand{\cardresource}[1]{
	\path[centerImg=#1] \shapeCard;
	\draw[cardcorners, thick, miniImage=#1] \shapeResourceMiniUL;
	\draw[cardcorners, thick,miniImage=#1] \shapeResourceMiniUR;
    \draw[cardcorners, thick, miniFlipImage=#1] \shapeResourceMiniBL;
    \draw[cardcorners, thick, miniFlipImage=#1] \shapeResourceMiniBR;
}

\newcommand{\cardRuleImg}[1]{
    % \node[anchor=south west,inner sep=0] at (0,0) {\includegraphics[width=\textwidth]{some_image.jpg}};
 	\path[ruleImage=#1] \shapeRuleImg;

}

% Inlined cardRuleImg data to slight raise the height above title strip
\pgfmathsetmacro{\BonusImgheight}{\cardheight-0.31*\cardheight-\stripwidth-2*\strippadding}
\newcommand{\cardBonusImg}[1]{
 	\path[
 	    path picture={
            \node at (path picture bounding box.center) {
                \includegraphics[
                    height=\BonusImgheight cm,
                    width=\RuleImgwidth cm,
                    keepaspectratio]{#1}
            };
        }
 	] (2*\strippadding+\stripwidth,0.31*\cardheight+\strippadding) rectangle (\cardwidth-2*\strippadding-\stripwidth,\cardheight-\stripwidth-\strippadding);

}

%   Kategorie der Karte
\newcommand{\cardtype}[3]{
    %   First we fill the intersecting area
    %   The \clip command does not allow options, therefore 
    %   we have to use a scope to set the even odd rule.
    \begin{scope}[even odd rule]
        %   Define a clipping path. All paths outside shapeCard will
        %   be cut because the even odd rule is set.
        \clip[cardcorners] \shapeCard;
        % Fill shapeLeftStripLong and shapeLeftStripShort.
        %   Since the even odd rule is set, only the card will be filled.
        \fill[strip,#1] \shapeLeftStripLong node[rotate=90,above left=0.9mm,font=\normalsize] {
            \color{white}\uppercase{#2}
        };
        \fill[strip,#1] \shapeLeftStripShort;
    \end{scope}

    \node at (\strippadding+\stripwidth-0.28,\cardheight-\strippadding-\strippadding-0.37) {\color{white}#3};
}
\newcommand{\cardtypeCharacter}{\cardtype{characterbg}{Charaktereigenschaft}
	{\hspace{-1mm}\huge\ding{78}}}

\newcommand{\cardtypeAbility}{\cardtype{abilitybg}{Fähigkeit}
	{\hspace{-1mm}\huge\ding{78}}}

\newcommand{\cardtypeItem}{\cardtype{itembg}{Gegenstand}
	{\hspace{-1mm}\LARGE bomb}}

\newcommand{\cardtypeTest}{\cardtype{testbg}{Testkarte}
	{\hspace{-1.4mm}\huge\ding{78}}}


%   Card Categories
\newcommand{\catancardtype}[2]{
    %   First we fill the intersecting area
    %   The \clip command does not allow options, therefore 
    %   we have to use a scope to set the even odd rule.
    \begin{scope}[even odd rule]
        %   Define a clipping path. All paths outside shapeCard will
        %   be cut because the even odd rule is set.
        \clip[cardcorners] \shapeCard;
        % Fill shapeLeftStripLong and shapeLeftStripShort.
        %   Since the even odd rule is set, only the card will be filled.
        % \fill[strip,#1] \shapeLeftStripLong node[rotate=90,above left=0.9mm,font=\normalsize] {
        %     \color{black}\uppercase{#2}
        % };
        \draw[strip,titlebg, thick, fill=#1] \shapeLeftStripExtra node[rotate=90,above left=0.9mm,font=\normalsize] {
            \color{black}\uppercase{#2}
        };
        % \draw[strip,titlebg, thick, fill=#1] \shapeLeftStripShort;
        % \draw[strip,titlebg, thick] \shapeRuleImg;
    \end{scope}

    % \node at (\strippadding+\stripwidth-0.28,\cardheight-\strippadding-\strippadding-0.37) {\color{black}#3};
}
\newcommand{\cardtypeRule}[1]{\catancardtype{rulebg}{Rules\quad ---\quad #1}}
\newcommand{\cardtypeBonus}[1]{\catancardtype{rulebg}{Bonus card\quad ---\quad #1}}

%   Titel der Karte
%\newcommand{\cardtitle}[1]{
%%    \draw[pattern=soft crosshatch,rounded corners=0.1cm] \shapeTitleArea;
%    \fill[elementcorners,titlebg,opacity=.85] \shapeTitleArea;
%    \node[text width=3.75cm] at (0.5*\cardwidth,\titley) {
%        \begin{center}
%            \color{white}\uppercase{\normalsize #1}
%        \end{center}
%    };
%}

\newcommand{\cardtitle}[1]{
	%    \draw[pattern=soft crosshatch,rounded corners=0.1cm] \shapeTitleArea;
	\draw[elementcorners,titlebg, ultra thick, fill=white] \shapeTitleArea;
	\node[text width=3.75cm] at (0.5*\cardwidth,\titley) {
		\begin{center}
			\color{black}\uppercase{\normalsize #1}
		\end{center}
	};
}

%   Inhalt der Karte
% Allows a divided section for game-text and lore-text
% Commented out since my cards use carddescriptor
% \newcommand{\cardcontent}[2]{
%     \begin{scope}[even odd rule]
%         \clip[cardcorners] \shapeCard;
%         \fill[elementcorners,contentbg] \shapeContentArea;
%     \end{scope}
%     \node[below right, text width=(\cardwidth-2*\strippadding-\stripwidth-2*\textpadding-0.3)*1cm] at (2*\strippadding+\stripwidth+\textpadding,0.5*\cardheight-\textpadding) {
%         \textit{\glqq\normalsize #1\grqq}
%     };
%     \node[below right, text width=(\cardwidth-2*\strippadding-\stripwidth-2*\textpadding-0.3)*1cm] at (2*\strippadding+\stripwidth+\textpadding,3) {
%         \vrule width \textwidth height 2pt \\[-2pt]
%         \vspace{0.2cm}
%         {\scriptsize #2}
%     };
% }




\newcommand{\carddescriptor}[1]{
	\begin{scope}[even odd rule]
		\clip[cardcorners] \shapeCard;
		\fill[elementcorners,textbg] \shapeDescriptorArea;
	\end{scope}
%	\node[below right, text width=(\cardwidth-2*\strippadding-\stripwidth-2*\textpadding-0.3)*1cm] at (2*\strippadding+\stripwidth+\textpadding,0.5*\cardheight-\textpadding) {
%		\textit{\glqq\normalsize #1\grqq}
%	};
	\node[below right, text width=(\cardwidth-2*\strippadding-2*\stripwidth-2*\textpadding-2*0.3)*1cm] at (2*\strippadding+\stripwidth+\textpadding,4) {
		\vrule width \textwidth height 2pt \\[-2pt]
		\vspace{0.4cm}
		{\large #1}
	};
}

\newcommand{\fulltikznode}[1]{
	\node
	[below right, text width=(\cardwidth-2*\strippadding-2*\stripwidth-2*\textpadding-2*0.3)*1cm] 
	at (2*\strippadding+\stripwidth+\textpadding,11) 
	{#1};
}

%% Deprecated
\newcommand{\depexpansion}{
%	\begin{scope}[even odd rule]
%		\clip[cardcorners] \shapeCard;
%		\clip[cardcorners] \shapeExpansionArea;
%%		\fill[strip,pricebg] \shapeRightStripShort;
%		\fill[green] \shapeTitleArea;
%%		\draw[elementcorners,titlebg, ultra thick,opacity=.85] \shapeExpansionArea;
%	\end{scope}
	\draw[elementcorners,titlebg, ultra thick,opacity=.85] \shapeExpansionArea;
	\node at (\cardwidth-0.5*\stripwidth-\strippadding-0.1,\titley-0.1) {\color{black}5-6};
}

\newcommand{\expansion}{
	\draw[cardcorners, titlebg, thick,opacity=.85]
	    (0.5*\cardwidth-0.75*\stripwidth,\strippadding) rectangle (0.5*\cardwidth+0.75*\stripwidth,\strippadding+\stripwidth);
\node at (0.5*\cardwidth,\strippadding+0.5*\stripwidth) {\scriptsize\color{black}5-6};
}


%   Preis der Karte
\newcommand{\cardprice}[1]{
    \begin{scope}[even odd rule]
        \clip[cardcorners] \shapeCard;
        \fill[strip,pricebg] \shapeRightStripShort;
    \end{scope}
    \node at (\cardwidth-0.5*\stripwidth-\strippadding,\titley-0.1) {\color{black}#1};
}